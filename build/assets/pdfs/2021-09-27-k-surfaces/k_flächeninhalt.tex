%%% Template originaly created by Karol Kozioł (mail@karol-koziol.net) and modified for ShareLaTeX use

\documentclass[a4paper,11pt]{article}

\usepackage{pdfpages}
\usepackage[german]{babel}
\usepackage[hyphens]{url} % break long url's with hyphens
\usepackage{hyperref}
\usepackage[utf8]{inputenc}
\usepackage{xcolor}
\usepackage{fancyhdr}

\usepackage{tocbibind}

%\usepackage{multicol}
\usepackage[breakable]{tcolorbox}
\usepackage{csquotes}

\usepackage{caption}
\usepackage{subcaption}

\usepackage{float}

\usepackage{footmisc} % footnote references

\usepackage{amsmath}
\usepackage{amssymb}
\usepackage{amsfonts}
\usepackage{amsthm}
\usepackage{mathtools}
\usepackage{svg}

%\usepackage{quiver}
\usepackage{tikz}
\usetikzlibrary{cd}
\usetikzlibrary{babel}

\usepackage{epigraph}
\setlength{\epigraphwidth}{.666\textwidth}

\usepackage{enumerate}
\usepackage[shortlabels]{enumitem}

%\usepackage[numbib]{tocbibind}

\usepackage[]{algorithm2e}

\usepackage[style=alphabetic]{biblatex}
\addbibresource{sources.bib}

\pagestyle{fancy} %eigener Seitenstil
\fancyhf{} %alle Kopf- und Fußzeilenfelder bereinigen
\fancyhead[L]{\leftmark} %Kopfzeile links mit Section Name. Mit \rightmark geht Subsection Name
\fancyhead[C]{} %zentrierte Kopfzeile
\fancyhead[R]{Stefan Volz} %Kopfzeile rechts
\renewcommand{\headrulewidth}{0.4pt} %obere Trennlinie
\fancyfoot[C]{\thepage} %Seitennummer
\renewcommand{\footrulewidth}{0.4pt} %untere Trennlinie

\setcounter{tocdepth}{2}
\setlength{\parindent}{0pt}

% custom theorem style für newlines zu beginn des Satzes
\newtheoremstyle{mainTheoremStyle}%    <name>
{\topsep}%   <space above>
{\topsep}%   <space below>
{}%  <body font> % was \itshape
{}%          <indent amount>
{\bfseries}% <Theorem head font>
{.}%         <punctuation after theorem head>
{\newline}%  <space after theorem head> (default .5em)
{}%          <Theorem head spec>
\theoremstyle{mainTheoremStyle}

% internal / raw theorem environments
\newtheorem{int_theorem}{Satz}[section]
\newtheorem{int_definition}[int_theorem]{Definition}
\newtheorem{notation}[int_theorem]{Notation}
\newtheorem{int_lemma}[int_theorem]{Lemma}
\newtheorem{int_corollary}[int_theorem]{Korollar}
\newtheorem{int_remark}[int_theorem]{Bemerkung}
\newtheorem{int_example}[int_theorem]{Beispiel}

\newenvironment{theorem}[1][]{%        % Create new environment which wraps our Theorem into a tcolorbox.
\begin{tcolorbox}[
    breakable,
    colback=blue!5!white,%     Background color.
    width=\dimexpr\linewidth+10pt\relax,%     Allow your box to be bigger than \linewidth ...
    enlarge left by=-5pt,%                    ... in order to have the text properly aligned. ...
    enlarge right by=-5pt,%                   ... Note that boxsep = -enlargeLeft = -enlargeRight = 0.5*enlargement of width. ...
    boxsep=5pt,%                              ... This is necessary to keep everything good looking.
    left=0pt,%                                Avoid extra space on the left, ...
    right=0pt,%                               ... right, ...
    top=0pt,%                                 ... top, ...
    bottom=0pt,%                              ... and bottom.
    arc=0pt,%                                 Corners not rounded.
    boxrule=0pt,%                             No boxrule.
    colframe=white]{}{}%                      Make rest of the boxrule invisible.
\ifstrempty{#1}{%                         If you didn't specify the optional argument of Theorem ...
    \begin{int_theorem}%                     ... then open a normal Theorem ...

}{%                                       ... else ...
    \begin{int_theorem}[#1]%                 ... open a Theorem and use the optional argument.
        }%
        }{%
    \end{int_theorem}%                          Close every environment.
    \end{tcolorbox}%
}

\newenvironment{definition}[1][]{%        % Create new environment which wraps our Theorem into a tcolorbox.
\begin{tcolorbox}[
    breakable,
    colback=red!5!white,%     Background color.
    width=\dimexpr\linewidth+10pt\relax,%     Allow your box to be bigger than \linewidth ...
    enlarge left by=-5pt,%                    ... in order to have the text properly aligned. ...
    enlarge right by=-5pt,%                   ... Note that boxsep = -enlargeLeft = -enlargeRight = 0.5*enlargement of width. ...
    boxsep=5pt,%                              ... This is necessary to keep everything good looking.
    left=0pt,%                                Avoid extra space on the left, ...
    right=0pt,%                               ... right, ...
    top=0pt,%                                 ... top, ...
    bottom=0pt,%                              ... and bottom.
    arc=0pt,%                                 Corners not rounded.
    boxrule=0pt,%                             No boxrule.
    colframe=white]{}{}%                      Make rest of the boxrule invisible.
\ifstrempty{#1}{%                         If you didn't specify the optional argument of Theorem ...
    \begin{int_definition}%                     ... then open a normal Theorem ...
}{%                                       ... else ...
    \begin{int_definition}[#1]%                 ... open a Theorem and use the optional argument.
        }%
        }{%
    \end{int_definition}%                          Close every environment.
    \end{tcolorbox}%
}

\newenvironment{lemma}[1][]{%        % Create new environment which wraps our Theorem into a tcolorbox.
\begin{tcolorbox}[
    breakable,
    colback=yellow!5!white,%     Background color.
    width=\dimexpr\linewidth+10pt\relax,%     Allow your box to be bigger than \linewidth ...
    enlarge left by=-5pt,%                    ... in order to have the text properly aligned. ...
    enlarge right by=-5pt,%                   ... Note that boxsep = -enlargeLeft = -enlargeRight = 0.5*enlargement of width. ...
    boxsep=5pt,%                              ... This is necessary to keep everything good looking.
    left=0pt,%                                Avoid extra space on the left, ...
    right=0pt,%                               ... right, ...
    top=0pt,%                                 ... top, ...
    bottom=0pt,%                              ... and bottom.
    arc=0pt,%                                 Corners not rounded.
    boxrule=0pt,%                             No boxrule.
    colframe=white]{}{}%                      Make rest of the boxrule invisible.
\ifstrempty{#1}{%                         If you didn't specify the optional argument of Theorem ...
    \begin{int_lemma}%                     ... then open a normal Theorem ...
}{%                                       ... else ...
    \begin{int_lemma}[#1]%                 ... open a Theorem and use the optional argument.
        }%
        }{%
    \end{int_lemma}%                          Close every environment.
    \end{tcolorbox}%
}

\newenvironment{corollary}[1][]{%        % Create new environment which wraps our Theorem into a tcolorbox.
\begin{tcolorbox}[
    breakable,
    colback=purple!5!white,%     Background color.
    width=\dimexpr\linewidth+10pt\relax,%     Allow your box to be bigger than \linewidth ...
    enlarge left by=-5pt,%                    ... in order to have the text properly aligned. ...
    enlarge right by=-5pt,%                   ... Note that boxsep = -enlargeLeft = -enlargeRight = 0.5*enlargement of width. ...
    boxsep=5pt,%                              ... This is necessary to keep everything good looking.
    left=0pt,%                                Avoid extra space on the left, ...
    right=0pt,%                               ... right, ...
    top=0pt,%                                 ... top, ...
    bottom=0pt,%                              ... and bottom.
    arc=0pt,%                                 Corners not rounded.
    boxrule=0pt,%                             No boxrule.
    colframe=white]{}{}%                      Make rest of the boxrule invisible.
\ifstrempty{#1}{%                         If you didn't specify the optional argument of Theorem ...
    \begin{int_corollary}%                     ... then open a normal Theorem ...
}{%                                       ... else ...
    \begin{int_corollary}[#1]%                 ... open a Theorem and use the optional argument.
        }%
        }{%
    \end{int_corollary}%                          Close every environment.
    \end{tcolorbox}%
}

\newenvironment{remark}[1][]{%        % Create new environment which wraps our Theorem into a tcolorbox.
\begin{tcolorbox}[
    breakable,
    colback=green!5!white,%     Background color.
    width=\dimexpr\linewidth+10pt\relax,%     Allow your box to be bigger than \linewidth ...
    enlarge left by=-5pt,%                    ... in order to have the text properly aligned. ...
    enlarge right by=-5pt,%                   ... Note that boxsep = -enlargeLeft = -enlargeRight = 0.5*enlargement of width. ...
    boxsep=5pt,%                              ... This is necessary to keep everything good looking.
    left=0pt,%                                Avoid extra space on the left, ...
    right=0pt,%                               ... right, ...
    top=0pt,%                                 ... top, ...
    bottom=0pt,%                              ... and bottom.
    arc=0pt,%                                 Corners not rounded.
    boxrule=0pt,%                             No boxrule.
    colframe=white]{}{}%                      Make rest of the boxrule invisible.
\ifstrempty{#1}{%                         If you didn't specify the optional argument of Theorem ...
    \begin{int_remark}%                     ... then open a normal Theorem ...
}{%                                       ... else ...
    \begin{int_remark}[#1]%                 ... open a Theorem and use the optional argument.
        }%
        }{%
    \end{int_remark}%                          Close every environment.
    \end{tcolorbox}%
}

\newenvironment{example}[1][]{%        % Create new environment which wraps our Theorem into a tcolorbox.
\begin{tcolorbox}[
    breakable,
    colback=yellow!5!white,%     Background color.
    width=\dimexpr\linewidth+10pt\relax,%     Allow your box to be bigger than \linewidth ...
    enlarge left by=-5pt,%                    ... in order to have the text properly aligned. ...
    enlarge right by=-5pt,%                   ... Note that boxsep = -enlargeLeft = -enlargeRight = 0.5*enlargement of width. ...
    boxsep=5pt,%                              ... This is necessary to keep everything good looking.
    left=0pt,%                                Avoid extra space on the left, ...
    right=0pt,%                               ... right, ...
    top=0pt,%                                 ... top, ...
    bottom=0pt,%                              ... and bottom.
    arc=0pt,%                                 Corners not rounded.
    boxrule=0pt,%                             No boxrule.
    colframe=white]{}{}%                      Make rest of the boxrule invisible.
\ifstrempty{#1}{%                         If you didn't specify the optional argument of Theorem ...
    \begin{int_example}%                     ... then open a normal Theorem ...
}{%                                       ... else ...
    \begin{int_example}[#1]%                 ... open a Theorem and use the optional argument.
        }%
        }{%
    \end{int_example}%                          Close every environment.
    \end{tcolorbox}%
}

%\usepackage{thmtools}
%\renewcommand{\listtheoremsname}{Sätze und Definitionen}

\newcommand{\underbraced}[2]{\ensuremath{\underset{#2}{\underbrace{#1}}}}

\renewcommand{\Re}{\text{Re}}
\renewcommand{\Im}{\text{Im}}

\def\R{\ensuremath{\mathbb{R}}}
\def\C{\ensuremath{\mathbb{C}}}
\def\N{\ensuremath{\mathbb{N}}}
\def\Z{\ensuremath{\mathbb{Z}}}
\def\H{\ensuremath{\mathbb{H}}}
\def\K{\ensuremath{\mathbb{K}}}
\def\e{e}
\def\i{\mathscr{i}}

\newenvironment{smallpmatrix}
{\left(\begin{smallmatrix}}
        {\end{smallmatrix}\right)}

\def\vol#1#2{\text{vol}_{#1}(#2)}
\def\Jacobian#1{J_{#1}}
\def\Rang#1{\text{Rang}(#1)}
\def\Innerproduct#1#2{\langle#1, #2 \rangle}
\def\LinearMaps#1#2{L(#1, #2)}
\def\diag{\text{diag}}
\def\norm#1{\lVert#1\rVert}
\def\shortbio#1#2#3{\textsc{#1}, #2, #3}
\def\floor#1{\left\lfloor{#1}\right\rfloor}
\def\trace#1{\text{Spur }#1}
\def\length#1{\mathcal{L}\ifstrempty{#1}{}{[#1]}}
\def\diff#1{\mathsf{D} #1}
\def\diffMulti#1#2{\mathsf{D}^{#2} #1}
\def\variation#1#2#3{\delta #1[#2, #3]}
\let\eps\epsilon
\let\epsilon\varepsilon
\def\orthogonal#1{#1^\bot}
\def\tangential#1{#1^\top}
\let\tilde\widetilde
\def\endomorphism#1{\text{End}(#1)}
\def\grad#1{\nabla#1}
\def\CovariantDerivative#1#2{\frac{\diff #1}{d#2}}


%%%----------%%%----------%%%----------%%%----------%%%

\begin{document}

\title{Rechtfertigung für Gramsche Determinante in Definition des $k$-Flächeninhalts}

\author{Stefan Volz}

\date{\today}

\maketitle

\section{Test}

\begin{definition}[$k$-Fläche]

    Sei $k \leq n \in \N$.
    Wir nennen $S \subseteq \R^n$ eine $k$-Fläche wenn
    \begin{itemize}
        \item $M$ eine $k$-dimensionale Mannigfaltigkeit ist und
        \item $S$ eine globale Parametrisierung besitzt.
    \end{itemize}
\end{definition}

Sei nun $S \subseteq \R^n$ eine $k$-Fläche mit Parametrisierung $\varphi : K \to S$.

Für $k = n$ wissen wir bereits, dass durch
\begin{align*}
    \vol{n}{S} = \int_K |\det \Jacobian{\varphi}(x)| d^n x
\end{align*}
ein Volumenbegriff definiert wird. Intuitiv misst hierbei die Funktionaldeterminante in jedem Punkt $x \in K$ wie sehr die Abbildung $\varphi$ den Raum $K$ ausdehnt um ihn zu $S$ zu verformen.

Für $k < n$ ist allerdings $\Jacobian{\varphi}(x) \in \R^{n, m}$ keine quadratische Matrix und somit können wir nicht die Determinante nutzen um diese Ausdehnung zu bestimmen, wir wollen daher eine allgemeinere Definition finden. Betrachten wir nun $\Jacobian{\varphi}(x)$ als lineare Abbildung $\ell : \R^k \to \R^n$ welche unsere Fläche verformt und in den höherdimensionalen Raum legt. Intuitiv könnten wir uns vorstellen dass wir diese Abbildung in zwei Teilabbildungen zerlegen können:
\begin{itemize}
    \item eine (lineare) Abbildung $\psi : \R^k \to \R^k$ welche den Verformungsanteil enthält und
    \item eine (lineare) Abbildung $\pi : \R^k \to \R^n$ welche unsere Fläche im $\R^n$ ablegt.
\end{itemize}
Sind $A_\psi \in \R^{k, k}$ und $A_\pi \in \R^{n,k}$ darstellende Matrizen dieser Abbildungen sollte also
\begin{align*}
    \ell = \pi \circ \psi \iff \Jacobian{\varphi}(x) = A_\pi A_\psi
\end{align*}
gelten. Welche Eigenschaften sollten diese Matrizen haben? Da $S$ eine $k$-dimensionale Untermannigfaltigkeit ist, gilt $\Rang{\Jacobian{\varphi}(x)} = k$. Die Rangungleich von Sylvester besagt, dass für $A \in \R^{n,k}, B \in \R^{k, m}$ folgende Ungleichung erfüllt ist:
\begin{align*}
    \Rang{A} + \Rang{B} - k \leq \Rang{AB} \leq \min{\Rang{A}, \Rang{B}}.
\end{align*}
Wir finden in unserem Fall also
\begin{align*}
    \Rang{A_\pi} + \Rang{A_\psi} - k \leq \Rang{\Jacobian{\varphi}(x)} = k \leq \min (\Rang{A_\pi}, \Rang{A_\psi}),
\end{align*}
woraus wir $\Rang{A_\pi} = \Rang{A_\psi} = k$ schließen. Wenden wir uns nun $A_\pi$ im Detail zu. Wie bereits erwähnt fordern wir, dass es unsere Fläche nicht weiter verformt, sondern nur im $R^n$ ablegt - wie lässt sich dies mathematisch präzisieren? Wir könnten uns z.B. vorstellen, dass die Form sich nicht verändert wenn alle Längen und Winkel bei der Abbildung erhielten blieben.
Der Winkel $\alpha \in [0, \pi]$ zwischen Vektoren $v,w \in \R^k$ ist definiert durch
\begin{align*}
    \cos \alpha = \frac{\Innerproduct{v}{w}}{\norm{v}_2 \norm{w}_2},
\end{align*}
unsere Längen- und Winkeltreue bedeutet also, dass für alle $v, w \in \R^k$
\begin{align*}
    \norm{A_\pi v}_2                                                          & = \norm{v}_2 \text{ und}                            \\
    \frac{\Innerproduct{A_\pi v}{A_\pi w}}{\norm{A_\pi v}_2 \norm{A_\pi w}_2} & = \frac{\Innerproduct{v}{w}}{\norm{v}_2 \norm{w}_2}
\end{align*}
gelten soll. Wir sehen schnell ein, dass dies äquivalent zu
\begin{align*}
    \Innerproduct{A_\pi v}{A_\pi w} & = \Innerproduct{v}{w}
\end{align*}
ist. Dies ist die definierende Eigenschaft von orthogonalen Abbildungen.
\begin{definition}{(Orthogonale Abbildung)}

    Seien $V, W$ $\K$-Vektorräume mit Skalarprodukten $\langle\cdot,\cdot\rangle_V$ und $\langle\cdot,\cdot\rangle_W$. Dann nennen wir $f \in \LinearMaps{V}{W}$ orthogonal wenn für alle $v, w \in V$
    \begin{align}
        \langle v, w\rangle_V = \langle f(v), f(w)\rangle_W
    \end{align}
    gilt.
\end{definition}
Die darstellenden Matrizen orthogonaler Abbildungen $\R^n \to \R^n$ bezeichnen wir als orthogonale Matrizen und die Menge all dieser Matrizen mit $O(n)$. Diese Menge wird auch die orthogonale Gruppe genannt. Allgemeiner bezeichnen wir die Menge aller $n \times k$ Matrizen mit orthonormalen Spalten durch $O(n,k)$. Analog definieren wir $U(n)$ als unitäre Gruppe für Abbildungen $\C^n \to \C^n$, bzw. die Notation $U(n, k)$ für nichtquadratische unitäre Matrizen.

\begin{remark}
    Orthogonale Abbildungen auf euklidischen Räumen beschreiben Drehungen, Spiegelungen und deren Kompositionen.
\end{remark}

Aus der linearen Algebra ergibt sich zudem der folgende Satz.
\begin{theorem}(QR-Zerlegung)

    Sei $K \in \{\R, \C\}$. Für eine Matrix $A \in \K^{n, m}$ mit Rang $m$ existieren eine spaltenorthogonale Matrix $Q \in O(n, m)$ bzw. $U(n, m)$ sowie eine obere Dreiecksmatrix $R \in GL_m(\K)$ mit
    \begin{align}
        A = QR.
    \end{align}
\end{theorem}
Wir sehen, dass wir also eine QR-Zerlegung von $\Jacobian{\varphi}(x)$ suchen.
Seien nun $A_\pi, A_\psi$ so eine QR-Zerlegung, dann finden wir analog zum Fall $n=k$ dass $A_\psi$ die Fläche mit ihrer Determinante, also dem Produkt ihrere Eigenwerte streckt und intuitiv wird diese gestreckte Fläche durch $A_\pi$ nichtmehr verändert. Es verbleibt die Frage wie wir diesen Streckungsfaktor bestimmen und ob er überhaupt wohldefiniert ist: wenn $Q, R$ und $Q', R'$ jeweils eine QR-Zerlegung von $\Jacobian{\varphi}(x)$ bilden, gilt dann stets $|\det Q| = |\det Q'|$? Die Antwort hierauf liefert wieder die lineare Algebra mittels der Singulärwertzerlegung.
\begin{theorem}{(Singulärwertzerlegung - SVD)}\label{theo:svd}

    Sei $A \in \C^{n,m}, m \leq n$. Dann existieren $V \in U(n), W \in U(m)$, so dass
    \begin{align}
        A = V \Sigma W^H \text{ mit } \Sigma = \begin{smallpmatrix}
            \Sigma_r & 0_{r, m-r} \\
            0_{n-r, r} & 0_{n-r, m-r}
        \end{smallpmatrix} \in \R^{n,m}
    \end{align}
    wobei $\Sigma_r = \diag(\sigma_1 \geq \sigma_2 \geq ... \geq \sigma_r > 0), r = \Rang{A}$. Wir bezeichnen die Werte $\sigma_i$ als Singulärwerte von $A$.
\end{theorem}
Ein analoger Satz gilt für reelle Matrizen, hierbei sind $V$ und $W$ orthogonal und die hermitesch Transponierte kann mit einer regulär Transponierten ersetzt werden.
Für unsere Zwecke wichtig ist, dass die Singulärwerte eindeutig sind.

Wir wollen nun die Singulärwertzerlegung nutzen um eine QR-Zerlegung zu bestimmen. Sei daher $\Jacobian{\varphi}(x) = V \Sigma W^T$ wie im Satz \ref{theo:svd}, dann können wir $\Sigma \in \R^{n,k}$ wie folgt zerlegen
\begin{align*}
    \Sigma = \begin{smallpmatrix}
        I_m \\
        0_{n-m}
    \end{smallpmatrix} \Sigma_r.
\end{align*}
Dann ist mit $Q := V \begin{smallpmatrix}
        I_m \\
        0_{n-m}
    \end{smallpmatrix}, R := \Sigma_r W^T$ eine QR-Zerlegung von $\Jacobian{\varphi}(x)$ gegeben. Da $W^T \in O(k)$ gilt $|\det W^T| = 1$ und somit
\begin{align*}
    |\det R| = \left|\prod_{i=1}^k \sigma_i \right|.
\end{align*}

\begin{remark}
    Für quadratische Matrizen stimmen die Singulärwerte mit den Eigenwerten überein. Wir führen also eine strikte Verallgemeinerung unserer Volumensdefinition durch.
\end{remark}

Wir würden nun gerne vermeiden, immer die SVD bestimmen zu müssen.
Hierzu benötigen wir das folgende Lemma:
\begin{lemma}
    $Q \in O(n)$ ist genau dann orthogonal wenn $Q^T Q = I_n$.
\end{lemma}

Wir bemerken
\begin{align*}
    \Jacobian{\varphi}(x)^T \Jacobian{\varphi}(x) = (V \Sigma W^T)^T V \Sigma W^T          & = W \Sigma V^T V \Sigma W^T = W \Sigma^2 W^T \\
    \underset{|\det W| = 1}{\implies} \det (\Jacobian{\varphi}(x)^T \Jacobian{\varphi}(x)) & = \det \Sigma^2 = \prod_{i=1}^k \sigma_i^2.
\end{align*}

\begin{definition}{(Maßtensor, Gram'sche Matrix / Determinante)}

    Sei $S \subseteq \R^n$ eine $k$-Fläche mit globaler Parametrisierung $\varphi : K \to S$. Für alle $x \in K$ nennen wir
    \begin{align*}
        G^\varphi(x) := \Jacobian{\varphi}(x)^T \Jacobian{\varphi}(x)
    \end{align*}
    Gram'sche Matrix von $\varphi$.
    Die Abbildung
    \begin{align*}
        g^\varphi : K \to \R^{k, k}, x \mapsto G^\varphi(x)
    \end{align*}
    nennen wir Maßtensor über $S$ und $\det G^\varphi(x)$ die Gram'sche Determinante von $\varphi$ bei $x \in K$.
\end{definition}

Schlussendlich definieren wir damit unseren Flächeninhalt

\begin{definition}($k$-Flächeninhalt)

    Sei $S \subseteq \R^n$ eine $k$-Fläche mit globaler Parametrisierung $\varphi : K \to S$. Dann ist der $k$-Flächeninhalt bzw. das $k$-Volumen von $S$ definiert durch
    \begin{align}
        \vol{k}{S} := \int_K \sqrt{\det g^\varphi (x)} d^kx.
    \end{align}
\end{definition}

\begin{example}(Oberfläche einer Kugel)

    Betrachten wir die Oberfläche der Kugel $B_r(0)$ mit Radius $r > 0$ als 2-Fläche im $\R^3$. Diese besitzt die globale Parametrisierung (abgesehen von einer Nullmenge)
    \begin{align}
        \sigma : (0, \pi) \times (0, 2\pi), (\vartheta, \varphi) \mapsto r \begin{smallpmatrix}
            \cos \vartheta \cos \varphi \\
            \cos \vartheta \sin \varphi \\
            \sin \vartheta
        \end{smallpmatrix}.
    \end{align}
    Die Jacobimatrix bei $x := (\vartheta, \varphi)$ ist
    \begin{align*}
        \Jacobian{\sigma}(x) = \begin{smallpmatrix}
            -\sin \vartheta \cos \varphi & - \cos \vartheta \sin \varphi \\
            -\sin \vartheta \sin \varphi & \cos \vartheta \cos \varphi \\
            \cos \vartheta & 0
        \end{smallpmatrix}
    \end{align*}
    und die Gram'sche Matrix somit
    \begin{align*}
        G^\sigma(x) = \Jacobian{\sigma}(x)^T \Jacobian{\sigma}(x) = r^2 \begin{smallpmatrix}
            1 & 0 \\
            0 & 1 - \sin^2 \vartheta
        \end{smallpmatrix}.
    \end{align*}
    Wir erhalten daher für den Flächeninhalt
    \begin{align*}
        \vol{2}{B_r(0)} & = \int_0^\pi \int_0^{2 \pi} \sqrt{r^4(1 - \sin^2 \vartheta)} d\varphi d\vartheta \\
                        & = 2 \pi r^2 \int_0^\pi |\cos \vartheta| d\vartheta                               \\
                        & = 4 \pi r^2.
    \end{align*}
\end{example}
\end{document}
