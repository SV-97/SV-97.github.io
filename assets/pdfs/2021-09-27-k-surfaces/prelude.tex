\usepackage{pdfpages}
\usepackage[german]{babel}
\usepackage[hyphens]{url} % break long url's with hyphens
\usepackage{hyperref}
\usepackage[utf8]{inputenc}
\usepackage{xcolor}
\usepackage{fancyhdr}

\usepackage{tocbibind}

%\usepackage{multicol}
\usepackage[breakable]{tcolorbox}
\usepackage{csquotes}

\usepackage{caption}
\usepackage{subcaption}

\usepackage{float}

\usepackage{footmisc} % footnote references

\usepackage{amsmath}
\usepackage{amssymb}
\usepackage{amsfonts}
\usepackage{amsthm}
\usepackage{mathtools}
\usepackage{svg}

%\usepackage{quiver}
\usepackage{tikz}
\usetikzlibrary{cd}
\usetikzlibrary{babel}

\usepackage{epigraph}
\setlength{\epigraphwidth}{.666\textwidth}

\usepackage{enumerate}
\usepackage[shortlabels]{enumitem}

%\usepackage[numbib]{tocbibind}

\usepackage[]{algorithm2e}

\usepackage[style=alphabetic]{biblatex}
\addbibresource{sources.bib}

\pagestyle{fancy} %eigener Seitenstil
\fancyhf{} %alle Kopf- und Fußzeilenfelder bereinigen
\fancyhead[L]{\leftmark} %Kopfzeile links mit Section Name. Mit \rightmark geht Subsection Name
\fancyhead[C]{} %zentrierte Kopfzeile
\fancyhead[R]{Stefan Volz} %Kopfzeile rechts
\renewcommand{\headrulewidth}{0.4pt} %obere Trennlinie
\fancyfoot[C]{\thepage} %Seitennummer
\renewcommand{\footrulewidth}{0.4pt} %untere Trennlinie

\setcounter{tocdepth}{2}
\setlength{\parindent}{0pt}

% custom theorem style für newlines zu beginn des Satzes
\newtheoremstyle{mainTheoremStyle}%    <name>
{\topsep}%   <space above>
{\topsep}%   <space below>
{}%  <body font> % was \itshape
{}%          <indent amount>
{\bfseries}% <Theorem head font>
{.}%         <punctuation after theorem head>
{\newline}%  <space after theorem head> (default .5em)
{}%          <Theorem head spec>
\theoremstyle{mainTheoremStyle}

% internal / raw theorem environments
\newtheorem{int_theorem}{Satz}[section]
\newtheorem{int_definition}[int_theorem]{Definition}
\newtheorem{notation}[int_theorem]{Notation}
\newtheorem{int_lemma}[int_theorem]{Lemma}
\newtheorem{int_corollary}[int_theorem]{Korollar}
\newtheorem{int_remark}[int_theorem]{Bemerkung}
\newtheorem{int_example}[int_theorem]{Beispiel}

\newenvironment{theorem}[1][]{%        % Create new environment which wraps our Theorem into a tcolorbox.
\begin{tcolorbox}[
    breakable,
    colback=blue!5!white,%     Background color.
    width=\dimexpr\linewidth+10pt\relax,%     Allow your box to be bigger than \linewidth ...
    enlarge left by=-5pt,%                    ... in order to have the text properly aligned. ...
    enlarge right by=-5pt,%                   ... Note that boxsep = -enlargeLeft = -enlargeRight = 0.5*enlargement of width. ...
    boxsep=5pt,%                              ... This is necessary to keep everything good looking.
    left=0pt,%                                Avoid extra space on the left, ...
    right=0pt,%                               ... right, ...
    top=0pt,%                                 ... top, ...
    bottom=0pt,%                              ... and bottom.
    arc=0pt,%                                 Corners not rounded.
    boxrule=0pt,%                             No boxrule.
    colframe=white]{}{}%                      Make rest of the boxrule invisible.
\ifstrempty{#1}{%                         If you didn't specify the optional argument of Theorem ...
    \begin{int_theorem}%                     ... then open a normal Theorem ...

}{%                                       ... else ...
    \begin{int_theorem}[#1]%                 ... open a Theorem and use the optional argument.
        }%
        }{%
    \end{int_theorem}%                          Close every environment.
    \end{tcolorbox}%
}

\newenvironment{definition}[1][]{%        % Create new environment which wraps our Theorem into a tcolorbox.
\begin{tcolorbox}[
    breakable,
    colback=red!5!white,%     Background color.
    width=\dimexpr\linewidth+10pt\relax,%     Allow your box to be bigger than \linewidth ...
    enlarge left by=-5pt,%                    ... in order to have the text properly aligned. ...
    enlarge right by=-5pt,%                   ... Note that boxsep = -enlargeLeft = -enlargeRight = 0.5*enlargement of width. ...
    boxsep=5pt,%                              ... This is necessary to keep everything good looking.
    left=0pt,%                                Avoid extra space on the left, ...
    right=0pt,%                               ... right, ...
    top=0pt,%                                 ... top, ...
    bottom=0pt,%                              ... and bottom.
    arc=0pt,%                                 Corners not rounded.
    boxrule=0pt,%                             No boxrule.
    colframe=white]{}{}%                      Make rest of the boxrule invisible.
\ifstrempty{#1}{%                         If you didn't specify the optional argument of Theorem ...
    \begin{int_definition}%                     ... then open a normal Theorem ...
}{%                                       ... else ...
    \begin{int_definition}[#1]%                 ... open a Theorem and use the optional argument.
        }%
        }{%
    \end{int_definition}%                          Close every environment.
    \end{tcolorbox}%
}

\newenvironment{lemma}[1][]{%        % Create new environment which wraps our Theorem into a tcolorbox.
\begin{tcolorbox}[
    breakable,
    colback=yellow!5!white,%     Background color.
    width=\dimexpr\linewidth+10pt\relax,%     Allow your box to be bigger than \linewidth ...
    enlarge left by=-5pt,%                    ... in order to have the text properly aligned. ...
    enlarge right by=-5pt,%                   ... Note that boxsep = -enlargeLeft = -enlargeRight = 0.5*enlargement of width. ...
    boxsep=5pt,%                              ... This is necessary to keep everything good looking.
    left=0pt,%                                Avoid extra space on the left, ...
    right=0pt,%                               ... right, ...
    top=0pt,%                                 ... top, ...
    bottom=0pt,%                              ... and bottom.
    arc=0pt,%                                 Corners not rounded.
    boxrule=0pt,%                             No boxrule.
    colframe=white]{}{}%                      Make rest of the boxrule invisible.
\ifstrempty{#1}{%                         If you didn't specify the optional argument of Theorem ...
    \begin{int_lemma}%                     ... then open a normal Theorem ...
}{%                                       ... else ...
    \begin{int_lemma}[#1]%                 ... open a Theorem and use the optional argument.
        }%
        }{%
    \end{int_lemma}%                          Close every environment.
    \end{tcolorbox}%
}

\newenvironment{corollary}[1][]{%        % Create new environment which wraps our Theorem into a tcolorbox.
\begin{tcolorbox}[
    breakable,
    colback=purple!5!white,%     Background color.
    width=\dimexpr\linewidth+10pt\relax,%     Allow your box to be bigger than \linewidth ...
    enlarge left by=-5pt,%                    ... in order to have the text properly aligned. ...
    enlarge right by=-5pt,%                   ... Note that boxsep = -enlargeLeft = -enlargeRight = 0.5*enlargement of width. ...
    boxsep=5pt,%                              ... This is necessary to keep everything good looking.
    left=0pt,%                                Avoid extra space on the left, ...
    right=0pt,%                               ... right, ...
    top=0pt,%                                 ... top, ...
    bottom=0pt,%                              ... and bottom.
    arc=0pt,%                                 Corners not rounded.
    boxrule=0pt,%                             No boxrule.
    colframe=white]{}{}%                      Make rest of the boxrule invisible.
\ifstrempty{#1}{%                         If you didn't specify the optional argument of Theorem ...
    \begin{int_corollary}%                     ... then open a normal Theorem ...
}{%                                       ... else ...
    \begin{int_corollary}[#1]%                 ... open a Theorem and use the optional argument.
        }%
        }{%
    \end{int_corollary}%                          Close every environment.
    \end{tcolorbox}%
}

\newenvironment{remark}[1][]{%        % Create new environment which wraps our Theorem into a tcolorbox.
\begin{tcolorbox}[
    breakable,
    colback=green!5!white,%     Background color.
    width=\dimexpr\linewidth+10pt\relax,%     Allow your box to be bigger than \linewidth ...
    enlarge left by=-5pt,%                    ... in order to have the text properly aligned. ...
    enlarge right by=-5pt,%                   ... Note that boxsep = -enlargeLeft = -enlargeRight = 0.5*enlargement of width. ...
    boxsep=5pt,%                              ... This is necessary to keep everything good looking.
    left=0pt,%                                Avoid extra space on the left, ...
    right=0pt,%                               ... right, ...
    top=0pt,%                                 ... top, ...
    bottom=0pt,%                              ... and bottom.
    arc=0pt,%                                 Corners not rounded.
    boxrule=0pt,%                             No boxrule.
    colframe=white]{}{}%                      Make rest of the boxrule invisible.
\ifstrempty{#1}{%                         If you didn't specify the optional argument of Theorem ...
    \begin{int_remark}%                     ... then open a normal Theorem ...
}{%                                       ... else ...
    \begin{int_remark}[#1]%                 ... open a Theorem and use the optional argument.
        }%
        }{%
    \end{int_remark}%                          Close every environment.
    \end{tcolorbox}%
}

\newenvironment{example}[1][]{%        % Create new environment which wraps our Theorem into a tcolorbox.
\begin{tcolorbox}[
    breakable,
    colback=yellow!5!white,%     Background color.
    width=\dimexpr\linewidth+10pt\relax,%     Allow your box to be bigger than \linewidth ...
    enlarge left by=-5pt,%                    ... in order to have the text properly aligned. ...
    enlarge right by=-5pt,%                   ... Note that boxsep = -enlargeLeft = -enlargeRight = 0.5*enlargement of width. ...
    boxsep=5pt,%                              ... This is necessary to keep everything good looking.
    left=0pt,%                                Avoid extra space on the left, ...
    right=0pt,%                               ... right, ...
    top=0pt,%                                 ... top, ...
    bottom=0pt,%                              ... and bottom.
    arc=0pt,%                                 Corners not rounded.
    boxrule=0pt,%                             No boxrule.
    colframe=white]{}{}%                      Make rest of the boxrule invisible.
\ifstrempty{#1}{%                         If you didn't specify the optional argument of Theorem ...
    \begin{int_example}%                     ... then open a normal Theorem ...
}{%                                       ... else ...
    \begin{int_example}[#1]%                 ... open a Theorem and use the optional argument.
        }%
        }{%
    \end{int_example}%                          Close every environment.
    \end{tcolorbox}%
}

%\usepackage{thmtools}
%\renewcommand{\listtheoremsname}{Sätze und Definitionen}

\newcommand{\underbraced}[2]{\ensuremath{\underset{#2}{\underbrace{#1}}}}

\renewcommand{\Re}{\text{Re}}
\renewcommand{\Im}{\text{Im}}

\def\R{\ensuremath{\mathbb{R}}}
\def\C{\ensuremath{\mathbb{C}}}
\def\N{\ensuremath{\mathbb{N}}}
\def\Z{\ensuremath{\mathbb{Z}}}
\def\H{\ensuremath{\mathbb{H}}}
\def\K{\ensuremath{\mathbb{K}}}
\def\e{e}
\def\i{\mathscr{i}}

\newenvironment{smallpmatrix}
{\left(\begin{smallmatrix}}
        {\end{smallmatrix}\right)}

\def\vol#1#2{\text{vol}_{#1}(#2)}
\def\Jacobian#1{J_{#1}}
\def\Rang#1{\text{Rang}(#1)}
\def\Innerproduct#1#2{\langle#1, #2 \rangle}
\def\LinearMaps#1#2{L(#1, #2)}
\def\diag{\text{diag}}
\def\norm#1{\lVert#1\rVert}
\def\shortbio#1#2#3{\textsc{#1}, #2, #3}
\def\floor#1{\left\lfloor{#1}\right\rfloor}
\def\trace#1{\text{Spur }#1}
\def\length#1{\mathcal{L}\ifstrempty{#1}{}{[#1]}}
\def\diff#1{\mathsf{D} #1}
\def\diffMulti#1#2{\mathsf{D}^{#2} #1}
\def\variation#1#2#3{\delta #1[#2, #3]}
\let\eps\epsilon
\let\epsilon\varepsilon
\def\orthogonal#1{#1^\bot}
\def\tangential#1{#1^\top}
\let\tilde\widetilde
\def\endomorphism#1{\text{End}(#1)}
\def\grad#1{\nabla#1}
\def\CovariantDerivative#1#2{\frac{\diff #1}{d#2}}
